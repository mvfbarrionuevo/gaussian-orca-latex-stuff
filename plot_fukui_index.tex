\documentclass[12pt,a4paper]{article}

%------------------------------------------------------
%	Normal document packages
%------------------------------------------------------
\usepackage{graphicx}
\usepackage{xcolor}
\usepackage[brazil]{babel}
\usepackage[utf8]{inputenc}
\usepackage{fancyhdr}
\usepackage{microtype}
\usepackage{cancel}
\usepackage{enumitem}
\usepackage{mathtools}
\usepackage{amsmath}
\usepackage{amssymb}
\usepackage{ragged2e}
\usepackage{gensymb}
\usepackage{lscape}
\usepackage{rotating}
\usepackage{tikz}
\usepackage{geometry}
\usepackage[LGRgreek]{mathastext}
\usepackage[labelformat=empty]{caption}

%------------------------------------------------------
%	Document packages to compile graphs
%------------------------------------------------------
\usepackage{pgfplots}
\pgfplotsset{compat=1.10} % you can change 'width' to fit your graph according to your document
\usepgfplotslibrary{external}	% this helps to get things done in a short time
\tikzexternalize			% it helps the above .spy (lib)
%------------------------------------------------------

\begin{document}
%\oddsidemargin=-27pt
\thispagestyle{empty}
\begin{figure}
\vspace{-2cm}
\centering
\begin{turn}{90}
\begin{tikzpicture}
\begin{axis}[
    width=25cm, height=15cm,
    enlargelimits=false,
    x tick label style={rotate=60},
    title={<Name your plot>},
    xlabel={Atom label},
    ylabel={Fukui indice [$E_h$]},
    xlabel shift = 1 pt,
    xtick=data,
    %xmin=-1, xmax=42,
    ymin=-0.010, ymax=0.035,
    xticklabels={<your atom labels, example: Zn0, C1, C2, H4 ... >},
    ytick={<your y ticks>  
},
    y tick label style={
        /pgf/number format/.cd,
            fixed,
            fixed zerofill,
            precision=2,
        /tikz/.cd
    },
    legend style = {at={(1.01,1)},
	anchor = north west,
	fill=white,
	align=left},
    ymajorgrids=true,
    xmajorgrids=true,
    grid style=dashed,
]

 
\addplot+[sharp plot] coordinates {<your coordinates here, example: (0,0.15)(1,0.00)(2,0.03)(4,0.35) ... >}; \addlegendentry{<your legend here>}

\end{axis}
\end{tikzpicture}
\end{turn}
\end{figure}

\end{document}
